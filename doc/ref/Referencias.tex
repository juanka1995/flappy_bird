\documentclass[12zpt]{article}
\usepackage[spanish, es-tabla]{babel}
\usepackage[top=2cm, bottom=2cm,left=2cm, right=2cm]{geometry}
\usepackage[utf8x]{inputenc}
\usepackage{amsmath}
\usepackage{enumerate}
\usepackage{graphicx}
\graphicspath{ {images/} }
\usepackage{xcolor}
\usepackage{hyperref}
\hypersetup{
colorlinks=false,
linkbordercolor={white}
}

\usepackage{parskip}
\usepackage{listings}
\usepackage{courier}
\lstset{
	numbers=left, % Muestra el número de linea
	stepnumber=1, % Empieza por el 1
	basicstyle=\footnotesize\ttfamily, % Standardschrift
	%numbers=left,               % Ort der Zeilennummern
	numberstyle=\tiny,          % Stil der Zeilennummern
	%stepnumber=2,               % Abstand zwischen den Zeilennummern
	numbersep=5pt,              % Abstand der Nummern zum Text
	tabsize=2,                  % Groesse von Tabs
	extendedchars=true,         %
	breaklines=true,            % Zeilen werden Umgebrochen
	keywordstyle=\color{red},
		frame=b,         
	%        keywordstyle=[1]\textbf,    % Stil der Keywords
	%        keywordstyle=[2]\textbf,    %
	%        keywordstyle=[3]\textbf,    %
	%        keywordstyle=[4]\textbf,   \sqrt{\sqrt{}} %
	stringstyle=\color{white}\ttfamily, % Farbe der String
	showspaces=false,           % Leerzeichen anzeigen ?
	showtabs=false,             % Tabs anzeigen ?
	xleftmargin=17pt,
	framexleftmargin=17pt,
	framexrightmargin=5pt,
	framexbottommargin=4pt,
	%backgroundcolor=\color{lightgray},
	showstringspaces=false      % Leerzeichen in Strings anzeigen ?        
}
\lstloadlanguages{% Check Dokumentation for further languages ...
	   %[Visual]Basic
	   %Pascal
	   %C
	   %C++
	   %XML
	   %HTML
	   Java
}
  %\DeclareCaptionFont{blue}{\color{blue}} 

%\captionsetup[lstlisting]{singlelinecheck=false, labelfont={blue}, textfont={blue}}
\usepackage{caption}
\DeclareCaptionFont{white}{\color{white}}
\DeclareCaptionFormat{listing}{\colorbox[cmyk]{0.74, 0.22, 0.0,0.21}{\parbox{\textwidth}{\hspace{15pt}#1#2#3}}}
\captionsetup[lstlisting]{format=listing,labelfont=white,textfont=white, singlelinecheck=false, margin=0pt, font={bf,footnotesize}}

\newcommand{\img}[1]{
\noindent\makebox[\textwidth][c]{\includegraphics[width=1\textwidth,]{#1}}%
}
\usepackage{fancyhdr} 
\pagestyle{fancy}
\fancyhf{}
\fancyhead[L]{\footnotesize Universidad de Granada} %encabezado izquierda
\fancyhead[R]{\footnotesize ETSIIT}   % dereecha
\fancyfoot[R]{\footnotesize Sistemas Gráficos}  % Pie derecha
\fancyfoot[C]{\thepage}  % centro
\fancyfoot[L]{\footnotesize 2018/2019}  %izquierda
\renewcommand{\footrulewidth}{0.4pt}

\begin{document}

%%%%%%%%%%%%%%%%%%%%%%%%%%%%%%%%%% PORTADA %%%%%%%%%%%%%%%%%%%%%%%%%%%%%%%%%%%%%%%%%%%%
																					%%%
\begin{center}																		%%%
\newcommand{\HRule}{\rule{\linewidth}{0.5mm}}									%%%\left
 																					%%%
\begin{minipage}{0.48\textwidth} \begin{flushleft}
\includegraphics[scale = 0.42]{logo1.png}
\end{flushleft}\end{minipage}
\begin{minipage}{0.48\textwidth} \begin{flushright}
\includegraphics[scale = 0.35]{logo2.png}
\end{flushright}\end{minipage}

													 								%%%
\vspace*{-1.5cm}								%%%
																					%%%	
\textsc{\huge Universidad de Granada\\ \vspace{5px} ETSIIT}\\[1.5cm]	

\textsc{\LARGE Sistemas Gráficos}\\[1.5cm]													%%%

%%%
    																				%%%
 			\vspace*{1cm}																		%%%
																					%%%
\HRule \\[0.4cm]																	%%%
{ \huge \bfseries Referencias - Práctica 2}\\[0.4cm]	%%%
 																					%%%
\HRule \\[1.5cm]																	%%%
 																				%%%
																					%%%
\begin{minipage}{0.46\textwidth}													%%%
\begin{center} \large															%%%	
\textbf{Autores:} Guillermo Bueno Vargas\\
Juan Carlos Ruiz García,\\
\textbf{Emails:} guillergood@correo.ugr.es, jcarlosruiz95@correo.ugr.es,\\
\textbf{Githubs:} Guillergood, juanka1995\\ 
\end{center}																		%%%
\end{minipage}		
																	%%%
\vspace{7cm} 																				
\begin{center}																					
{\large \today}																	%%%
 			\end{center}												  						
\end{center}							 											
																					
\newpage																		
%%%%%%%%%%%%%%%%%%%% TERMINA PORTADA %%%%%%%%%%%%%%%%%%%%%%%%%%%%%%%%

%%%%%%%%%%%%%%%%%%%% INDICE %%%%%%%%%%%%%%%%%%%%%%%%%%%%%%%%

\begin{thebibliography}{9}
	\bibitem{doc} 
	Documentacion oficial de Three.js
	\url{https://threejs.org/docs/index.html}
	
	\bibitem{coin} 
	Shader para la moneda.
	\url{https://james.greenle.af/articles/coins}

	\bibitem{heart} 
	Shader para el corazón.
	\url{https://threejsfundamentals.org/threejs/lessons/threejs-primitives.html}

	\bibitem{callback} 
	Uso de callbacks.
	\url{https://developer.mozilla.org/es/docs/Glossary/Callback_function}
	
	\bibitem{box3_collisions} 
	Uso de Box3 para colisiones.
	\url{http://man.hubwiz.com/docset/HTML.docset/Contents/Resources/Documents/developer.mozilla.org/en-US/docs/Games/Techniques/3D_collision_detection/Bounding_volume_collision_detection_with_THREE.html},
	\url{https://threejs.org/docs/index.html#api/en/math/Box3}
\end{thebibliography}

%%%%%%%%%%%%%%%%%%%% TERMINA INDICE %%%%%%%%%%%%%%%%%%%%%%%%%%%%%%%%

% \lstinputlisting[label=samplecode,caption=A sample]{../E1/salida.txt}

% \noindent\makebox[\textwidth][c]{\includegraphics[width=0.25\textwidth,]{6.png}}%

\end{document}
